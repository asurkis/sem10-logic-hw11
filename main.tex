\documentclass{article}

\usepackage[T2A]{fontenc}
\usepackage[utf8]{inputenc}
\usepackage[russian]{babel}
\usepackage{fullpage}
\usepackage{indentfirst}

\usepackage{amsmath}
\usepackage{amsfonts}
\usepackage{amsthm}
\usepackage{float}
\usepackage{tikz}
\usepackage{proof}
\usepackage{mathtools}

%\usepackage[outputdir=out]{minted}
%\usepackage{algorithmicx}
%\usepackage{algpseudocodex}
%\usepackage{algorithm2e}
\usepackage{hyperref}
\usepackage{enumitem}
\usepackage{array}

\usepackage[outputdir=out]{minted}

%\setlength{\parindent}{1.25cm}
%\renewcommand{\baselinestretch}{1.5}
\setlength{\parskip}{6pt}

\begin{document}
    \theoremstyle{definition}
    \newtheorem*{definition}{Определение}
    \newtheorem{theorem}{Теорема}
    \newtheorem{statement}{Утверждение}
    \newtheorem{lemma}{Лемма}

    \renewcommand{\a}{\alpha}

    \newcommand{\paren}[1]{\left ( #1 \right )}
    \newcommand{\brackets}[1]{\left [ #1 \right ]}
    \newcommand{\braces}[1]{\left \{ #1 \right \}}
    \newcommand{\floor}[1]{\left \lfloor #1 \right \rfloor}
    \newcommand{\ceil}[1]{\left \lceil #1 \right \rceil}
    \newcommand{\abs}[1]{\left | #1 \right |}
    \newcommand{\fn}[2]{\text{#1}\paren{#2}}

    \hfill
    \begin{tabular}{ll}
        Студент: & Антон Суркис \\
        Группа:  & M4141        \\
        Дата:    & \today       \\
    \end{tabular}
    \hrule

    \section{Задача 1}
Напишите программу на любом удобном языке,
входящем в \url{https://pypl.github.io/PYPL.html},
которая печатает в стандартный вывод MD5-сумму файла,
в котором записана.
Проверить результат можно командой вроде (осторожно, башизм)
\begin{verbatim}
diff <(./program) <(md5sum program.source)
\end{verbatim}
Запрещено совершать ввод-вывод кроме печати результата.

\subsection{Решение}
\url{https://github.com/asurkis/sem10-logic-hw11/blob/main/code/src/bin/task1.rs}

В файле \texttt{code/src/bin/task1.rs}:

\inputminted{rust}{code/src/bin/task1.rs}

    \section{Задача 2}
Напишите такую функцию на любом удобном языке из PYPL, к
оторая принимает как аргумент число n и печатает в
стандартный вывод первые n символов файла,
в котором записана.

\subsection{Решение}
\url{https://github.com/asurkis/sem10-logic-hw11/blob/main/code/src/bin/task2.rs}

В файле \texttt{code/src/bin/task2.rs}:

\inputminted{rust}{code/src/bin/task2.rs}

    \section{Задача 3}
Пусть дана таблица бинарной логической операции:
\[
    \begin{array}{ccc}
        a & b & a \oplus b \\ \hline
        0 & 0 & n_0 \\
        0 & 1 & n_1 \\
        1 & 0 & n_2 \\
        1 & 1 & n_3 \\
    \end{array}
\]
Опишите конфигурацию машины Тьюринга,
которая принимает на вход биты $n_1$, $n_2$, $n_3$, $n_4$,
пробел, одно число в двоичной записи,
пробел, второе число в двоичной записи.
В случае несовпадения длин поданных чисел число
меньшей длины дополняется ведущими нулями.
Требуется найти результат побитового применения
заданной логической операции к поданным числам.
Например, вход 0110 0011 1010 означает,
что требуется найти xor чисел 3 и 10,
то есть результат должен быть 1001.

\subsection{Решение}
Алфавит: 1, 0, пустой символ.

Общая идея:
\begin{enumerate}
    \item Считаем $n_0, n_1, n_2, n_3$ и запишем их в состояние
    (т.к. их комбинаций всего 16)

    \item \label{task3:loop-start} Сдвинем первое число вправо на 1,
    откусим последний бит и запомним его в состоянии

    \item Откусим последний бит второго числа,
    справа от того, где он был, запишем соответствующий бит ответа

    \item Вернёмся к началу первого числа и повторим
    с пункта~\ref{task3:loop-start}

    \item Если кончилось первое число,
    то тривиальным проходом запишем на позиции
    на 1 правее остатка второго числа остаток ответа,
    затем перейдём к началу ответа

    \item Если кончилось второе число,
    то притянем остаток ответа из первого числа,
    окажемся на две клетки левее ответа,
    и перейдём на две клетки вправо
\end{enumerate}

Решением будет описание машины Тьюринга,
генерируемое следующим кодом на Rust
(также доступно по
\url{https://github.com/asurkis/sem10-logic-hw11/blob/main/code/src/bin/task2.rs}
или в файле \texttt{code/src/bin/task3.rs})
(само описание занимает 1672 строки, а код всего 139):
\inputminted{rust}{code/src/bin/task3.rs}

Примеры работы:

\begin{tabular}{ll}
    Ввод & Результат \\ \hline
    \texttt{0110\_0011\_1010} & \texttt{1001} \\
    \texttt{0110\_1010\_0011} & \texttt{1001} \\
    \texttt{1000\_11\_1010} & \texttt{0100} \\
    \texttt{0110\_11\_1010} & \texttt{1001} \\
    \texttt{0110\_1010\_11} & \texttt{1001} \\
    \texttt{1101\_1010\_11} & \texttt{0111} \\
\end{tabular}

\qed

    \section{Задача 4}
Опишите правила перехода для одноленточной машины Тьюринга,
которая вычитает одно число в унарной записи из другого.
Числа записаны подряд без пробела.
Например, если изначально на ленте находится $SSSSSOSSO$,
то после остановки машины на ленте справа от
курсора должно быть $SSSO$.
Машина должна принимать язык $\{S^{n+k} O S^n O\}$.

\subsection{Решение}
См. задачу 5.

\qed

\end{document}
