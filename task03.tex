\section{Задача 3}
Пусть дана таблица бинарной логической операции:
\[
    \begin{array}{ccc}
        a & b & a \oplus b \\ \hline
        0 & 0 & n_0 \\
        0 & 1 & n_1 \\
        1 & 0 & n_2 \\
        1 & 1 & n_3 \\
    \end{array}
\]
Опишите конфигурацию машины Тьюринга,
которая принимает на вход биты $n_1$, $n_2$, $n_3$, $n_4$,
пробел, одно число в двоичной записи,
пробел, второе число в двоичной записи.
В случае несовпадения длин поданных чисел число
меньшей длины дополняется ведущими нулями.
Требуется найти результат побитового применения
заданной логической операции к поданным числам.
Например, вход 0110 0011 1010 означает,
что требуется найти xor чисел 3 и 10,
то есть результат должен быть 1001.

\subsection{Решение}
Алфавит: 1, 0, пустой символ.

Общая идея:
\begin{enumerate}
    \item Считаем $n_0, n_1, n_2, n_3$ и запишем их в состояние
    (т.к. их комбинаций всего 16)

    \item \label{task3:loop-start} Сдвинем первое число вправо на 1,
    откусим последний бит и запомним его в состоянии

    \item Откусим последний бит второго числа,
    справа от того, где он был, запишем соответствующий бит ответа

    \item Вернёмся к началу первого числа и повторим
    с пункта~\ref{task3:loop-start}

    \item Если кончилось первое число,
    то тривиальным проходом запишем на позиции
    на 1 правее остатка второго числа остаток ответа,
    затем перейдём к началу ответа

    \item Если кончилось второе число,
    то притянем остаток ответа из первого числа,
    окажемся на две клетки левее ответа,
    и перейдём на две клетки вправо
\end{enumerate}

Решением будет описание машины Тьюринга,
генерируемое следующим кодом на Rust
(также доступно по
\url{https://github.com/asurkis/sem10-logic-hw11/blob/main/code/src/bin/task2.rs}
или в файле \texttt{code/src/bin/task3.rs})
(само описание занимает 1672 строки, а код всего 139):
\inputminted{rust}{code/src/bin/task3.rs}

Примеры работы:

\begin{tabular}{ll}
    Ввод & Результат \\ \hline
    \texttt{0110\_0011\_1010} & \texttt{1001} \\
    \texttt{0110\_1010\_0011} & \texttt{1001} \\
    \texttt{1000\_11\_1010} & \texttt{0100} \\
    \texttt{0110\_11\_1010} & \texttt{1001} \\
    \texttt{0110\_1010\_11} & \texttt{1001} \\
    \texttt{1101\_1010\_11} & \texttt{0111} \\
\end{tabular}
