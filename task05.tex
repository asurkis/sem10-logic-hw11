\section{Задача 5}
Выполните предыдущее задание, не расширяя алфавит новыми символами.

\subsection{Решение}
Считаем, что пустой символ --- $O$, т.к.
нам достаточно всего двух символов.
Если это не так, то просто добавим для $O$
такие же переходы, как для пустого символа.

Тогда на сайте \url{https://turingmachinesimulator.com/}
описанием нужной машины Тьюринга будет следующий код:
\begin{verbatim}
name: Unary subtraction
init: leftmost
accept: qfin

leftmost,_
qfin,_,-

leftmost,S
seekRight1,S,>

seekRight1,S
seekRight1,S,>

seekRight1,_
seekRight2,_,>

seekRight2,S
seekRight3,S,>

seekRight2,_
seekFin1,_,<

seekRight3,S
seekRight3,S,>

seekRight3,_
seekLeft1,_,<

seekLeft1,S
seekLeft2,_,<

seekLeft2,S
seekLeft2,S,<

seekLeft2,_
seekLeft3,_,<

seekLeft3,S
seekLeft3,S,<

seekLeft3,_
seekLeft4,_,>

seekLeft4,S
leftmost,_,>

seekFin1,_
seekFin2,_,<

seekFin2,S
seekFin2,S,<

seekFin2,_
qfin,_,>
\end{verbatim}

То есть сначала мы находимся в самой левой ненулевой ячейке первого числа,
ищем самую правую ненулевую ячейку второго числа.

Если такой нет (т.е. второе число --- 0),
то переходим до самой левой ячейки первого
числа, и первое число будет ответом.

Если такая есть, то зануляем её, ищем самую левую ненулевую
ячейку первого числа, зануляем её, и переходим на 1 вправо,
оба числа уменьшены на 1, мы принимаем вход заново.

Если первое число оказалось нулём, считаем,
что второе также будет нулём.

\qed
